% Plantilla de reportes del Laboratorio de Programacion de Sistemas
\documentclass[10pt,letterpaper,oneside]{article}
\usepackage[latin1]{inputenc}
\usepackage{longtable}
% sudo apt-get install texlive-lang-spanish 
\usepackage[spanish]{babel}

% Cambiar por el n\'umero y nombre de la pr\'actica correspondiente
\title{Pr\'actica 6 \\ Paso 2 del ensamblador para SIC extendida \\  Laboratorio de Programaci\'on de Sistemas}

% Cambiar por nombre del alumno correspondiente
\author{Jorge Armando Tovar Ojeda}

% Cambiar cada una de las secciones por las indicadas en la pr\'actica correspondiente
\begin{document}

\maketitle

\section{Objetivo}
Ensamblar las instrucciones de acuerdo a los formatos y modos de direccionamiento de la SIC extendida y generar el archivo objeto agregando los registros de modificacion correspondientes.


\section{Otras secciones}
\subsection{Explicar el mecanismo utilizado para calcular cada una de las banderas n, i, x, b, p, e y el desplazamiento/direccion.}
Para esta seccion se utilizaron enteros para representar las banderas de estado en donde: n,i se tomaron como 2 bits, x,p,e se tomaron como 1 bit y el desplazamiento como 12 bits en caso de el formato 3 y 20 bits en el formato 4 asi como se muestra a continuacion:
\\
\\
Assemblef3 Ensambla el formato 3\\
func (p *Assembler) Assemblef3(line int, opcode int64, ni int64, x int64, b int64, pp int64, e int64, desp int64) \\
\indent	opcode = opcode $|$ ni (se juntan los ultimos 2 bits de opcode con ni)\\
\indent	asmInst := desp $|$ (e\textless\textless12) $|$ (pp \textless\textless 13) $|$ (b \textless\textless 14) $|$ (x\textless\textless 15) $|$ (opcode \textless\textless 16)  \\
\indent	asmInstStr := strconv.FormatInt(asmInst, 16)\\
\indent	asmInstStr = strings.ToUpper(asmInstStr)\\
\indent asmInstStr = p.formaTo3byte(asmInstStr)\\
\indent	p.Hexcode = append(p.Hexcode, asmInstStr)\\
\indent	p.Addrs = append(p.Addrs, GetTabSim().Progpc[line])\\
\\
\\
\\
Assemblef4 Ensambla el formato 4\\
func (p *Assembler) Assemblef4(line int, opcode int64, ni int64, x int64, b int64, pp int64, e int64, desp int64) \\
	opcode = opcode $|$ ni\\
	asmInst := desp $|$ (e \textless\textless 20) $|$ (pp \textless\textless 21) $|$ (b \textless\textless 22) $|$ (x \textless\textless 23) $|$ (opcode \textless\textless 24)\\
	asmInstStr := strconv.FormatInt(asmInst, 16)\\
	asmInstStr = strings.ToUpper(asmInstStr)\\
	asmInstStr = p.formaTo4byte(asmInstStr)\\
	p.Hexcode = append(p.Hexcode, asmInstStr)\\
	p.Addrs = append(p.Addrs, GetTabSim().Progpc[line])\\
\\
Para el caso del desplazamiento se utilizo el algoritmo visto en clase respecto a la relatividad del desplazamiento con respecto al registro CP o al registro B como muestra el siguiente codigo: \\
\\
AssembleXE ensambla la sic extendida\\
func (p *Assembler) AssembleXE(line int, nemonic string, indexed int64, symbol string, bRegister int64, ni int64)  \\
	ta = GetTabSim().GetSymbolAddrs(symbol)\\
	desp = ta - GetTabSim().Progpc[line]\\
	opcode := p.GetCodeEntry(nemonic)\\
	if desp \textless= 2047 AND desp \textgreater= -2048\\ 
		if desp < 0  \\
			desp = 4096 + desp\\
		p.Assemblef3(line, opcode, ni, indexed, 0, 1, 0, desp)\\
	 else \\
		desp := ta - bRegister\\
		if desp \textgreater= 0 AND desp \textless= 4095 \\
			p.Assemblef3(line, opcode, ni, indexed, 1, 0, 0, desp)\\
		 else  \\
			p.Assemblef3(line, opcode, ni, indexed, 1, 1, 0, 4095)\\
		
	

\subsection{Explicar el mecanismo utilizado para almacenar y generar los registros de modificacion.}
En este caso los registros de modificacion se almacenan en un arreglo de strings, se generan cuando un formato 4 es ensamblado y el algoritmo implementado fue el siguiente:\\
\\
Se busca en la tabla de simbolos la direccion del desplazamiento\\
Si la encuentra significa que se tiene que hacer una nueva entrada de registro relocalizable\\
Se ensambla segun donde se encuentre el contador de programa + 1 con un desplazamiento de 5 medios bytes y un + (ya que siempre se le va a sumar mientras no haya terminos relativos negativos)\\
\\
for key, value = range GetTabSim().table \\
		num, err = strconv.ParseInt(asmInstStr, 16, 0)\\
		num = num AND 0x000FFFFF\\
		if value == num \\
			found = true\\
			if found \\
					dirstr = strings.ToUpper(p.formaTo3byte(strconv.FormatInt(GetTabSim().Progpc[line-1]+1, 16)))\\
					p.mReg = append(p.mReg, "M"+dirstr+"05"+"+")\\
				found = false\\
		
\subsection{Explicar el mecanismo utilizado para soportar la compatibilidad hacia atr\'as.}
Se tomo mucho en cuenta la extension del archivo para realizar la compatibilidad ya que el ensamblado de el direccionamiento simple tiene la misma sintaxis tanto para la SIC estandar como para la SICXE, el siguiente fragmento de codigo muestra como se realizo la validacion de la extension del archivo\\
\\
	simple3:\\
			NEMONICO identificador indexado \\
					if yylex.(*yylexer).firstparse == false AND yylex.(*yylexer).isXE == false	\\
						(Ensambla el la SIC ESTANDAR)\\
					 else\\
							if yylex.(*yylexer).firstparse == false AND yylex.(*yylexer).isXE == true	\\
								if format4 == false\\
								(Ensambla el la SIC XE con formato 3)\\
								else\\
								(Ensambla el la SIC XE con formato 4)\\				

Fragmento de codigo donde se le manda la bandera si es extendida o no\\
func (p *Parser) Parse(r io.Reader, flag bool, isXE bool) string \\
	l := newLexer(bufio.NewReader(r))\\
	l.firstparse = flag\\
	l.isXE = isXE\\
	yyParse(l)\\
	return l.str\\
\\

\subsection{Describir el procedimiento utilizado para generar el valor hexadecimal para los desplazamientos negativos en instrucciones de formato 3.}
Para este procedimiento se realizo el complemento a dos, si el numero es negativo se le suma 4096 de esta forma el numero queda listo para ser ensamblado.\\
if desp < 0 \\
	desp = 4096 + desp\\
		
\subsection{Describir los problemas que se hayan presentado durante la pr\'actica y como fueron solucionados.}
Se presentaron problemas a la hora de hacer la compatibilidad hacia atras ya que el analizador sintactico confunde entre la SIC estandar y la SICXE, se soluciono haciendo uso de banderas como se describio en la seccion anterior.\\

\subsection{Descripcion de las actividades realizadas por fecha y hora}

\begin{longtable}{| p{.20\textwidth} | p{.80\textwidth} |} 
	\hline
     	 Fecha Hora & Descripcion \\  
      	\hline
	\date{2015-11-05}  (0hrs)&  Practica creada \\
      	\hline
      	\hline
	\date{2015-11-08}  (8hrs)&  Se elaboro el mecanismo para manejar las banderas nixbpe y los formatos\\
      	\hline
      	\hline
	\date{2015-11-09}  (8hrs)& Se elaboro el codigo objeto\\
      	\hline
      	\hline
	\date{2015-11-12}  (3hrs)&  detalles y reporte\\
      	\hline
      	\hline
\caption{Actividades realizadas} % needs to go inside longtable environment
\label{tab:myfirstlongtable}
\end{longtable}

Total de tiempo: 19hrs

\section{Conclusiones y posibles mejoras}
Existen mejoras que pueden hacerse por ejemplo en la gramatica hay renundancias y por lo tanto el codigo se repite en ciertas partes, se puede realizar una gramatica mas optima, puedo concluir de esta practica demuestra la compatibilidad y la importancia de esta en los sistemas de computo actuales tales como la arquitectura x86 de 32 bits y la x64 de 64 bits de intel \\
Hora y dia de laboratorio externo: Lunes 7am-8am. 

\section{Referencias consultadas}
Se consulto solamente la documentacion de golang para el uso del mapa y la funcion que convierte a distintas bases. \\
http://golang.org/pkg/strconv/   \\
https://blog.golang.org/go-maps-in-action

\end{document}
